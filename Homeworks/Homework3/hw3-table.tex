\begin{table}[h]
        \centering
        \renewcommand{\arraystretch}{1.3}
        \setlength{\tabcolsep}{5pt}
        \resizebox{\textwidth}{!}{ % Automatically resizes the table to fit it on the pdf
        \begin{tabular}{|c|c|c|c|c|c|}
    \hline
\textbf{Galaxy Name} & \textbf{Halo Mass $(10^{12}M_\odot)$} & \textbf{Disk Mass $(10^{12}M_\odot)$} & \textbf{Bulge Mass $(10^{12}M_\odot)$} & \textbf{Total $(10^{12}M_\odot)$} & \textbf{$f_\text{bar}$} \\
    \hline
MW & 1.975 & 0.075 & 0.01 & 2.06 & 0.041 \\
    \hline
M31 & 1.921 & 0.12 & 0.019 & 2.06 & 0.067 \\
    \hline
M33 & 0.187 & 0.009 & 0.0 & 0.196 & 0.046 \\
    \hline
\end{tabular}
        }

        % ADD CAPTION (CHANGE IF COPY-PASTING)
        
        \caption{This table displays the total masses of the galactic
        halo, disk stars, and bulge stars in the Milky Way Galaxy
        (MW), Andromeda Galaxy (M31) and Triangulum Galaxy (M33). In
        addition, the total mass of each galaxy and the ratio of
        stellar mass to total mass, also known as the baryon
        fraction, is included in the fifth and sixth columns
        respectively. Note that the galactic halo is assumed to be
        entirely composed of dark matter. All values are in units
        $10^{12}\,M_\odot$.}
        
        \label{tab:galaxyMasses}
        \end{table}