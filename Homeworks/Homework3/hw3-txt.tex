\documentclass{article}

\usepackage{amsmath}
\usepackage{graphicx}
\usepackage[a4paper, margin=0.5in]{geometry} % Adjusts margin size

% \usepackage{natbib}

\title{Homework 3 — pdf portion} 
\author{Christian Burt\\ASTR 400B}
\date{February 6, 2025}

% The preamble ends with the command \begin{document}
\begin{document}
\maketitle
\setcounter{section}{2} % Section counter += 2
    
    \section{Table of Galactic Masses}
    
    \begin{table}[h]
        \centering
        \renewcommand{\arraystretch}{1.3}
        \setlength{\tabcolsep}{5pt}
        \resizebox{\textwidth}{!}{ % Automatically resizes the table to fit it on the pdf
        \begin{tabular}{|c|c|c|c|c|c|}
    \hline
\textbf{Galaxy Name} & \textbf{Halo Mass $(10^{12}M_\odot)$} & \textbf{Disk Mass $(10^{12}M_\odot)$} & \textbf{Bulge Mass $(10^{12}M_\odot)$} & \textbf{Total $(10^{12}M_\odot)$} & \textbf{$f_\text{bar}$} \\
    \hline
MW & 1.975 & 0.075 & 0.01 & 2.06 & 0.041 \\
    \hline
M31 & 1.921 & 0.12 & 0.019 & 2.06 & 0.067 \\
    \hline
M33 & 0.187 & 0.009 & 0.0 & 0.196 & 0.046 \\
    \hline
\end{tabular}
        }

        % ADD CAPTION (CHANGE IF COPY-PASTING)
        
        \caption{This table displays the total masses of the galactic
        halo, disk stars, and bulge stars in the Milky Way Galaxy
        (MW), Andromeda Galaxy (M31) and Triangulum Galaxy (M33). In
        addition, the total mass of each galaxy and the ratio of
        stellar mass to total mass, also known as the baryon
        fraction, is included in the fifth and sixth columns
        respectively. Note that the galactic halo is assumed to be
        entirely composed of dark matter. All values are in units
        $10^{12}\,M_\odot$.}
        
        \label{tab:galaxyMasses}
        \end{table}
    
    \section{Questions}

    \begin{enumerate}
        \item MW and M31 have masses that differ by $\le \, 10^{10} M_\odot$
          or 1\% of their respective total masses according to this simulation
          
        \item Per Table \ref{tab:galaxyMasses}, MW has a total stellar mass of
          $0.085 \cdot 10^{12} M_\odot$ and M31 has a total stellar mass of
          $0.139 \cdot 10^{12} M_\odot$, meaning M31 has a greater stellar
          mass. The relation between stellar mass and luminosity is very
          intricate and not fully known, but is broadly true that a larger
          stellar mass corresponds to a larger stellar luminosity. Therefore,
          we can say that M31 is also more luminous.

        \item Per Table \ref{tab:galaxyMasses}, MW has a total halo mass of
          $1.975 \cdot 10^{12} M_\odot$ and M31 has a total stellar mass of
          $1.921 \cdot 10^{12} M_\odot$, meaning MW has a $\sim3\%$ greater
          mass of dark matter. It is not immediately intuitive why M31 has a
          larger stellar mass but MW has a larger halo mass, and I would
          suspect the answer to this question lies in the history of mergers
          both galaxies experienced.

        \item The baryon fraction of both galaxies are signficantly lower than
          the baryon fraction of the universe. I suspect this is due to
          dissipation of gas and ejected stars throughout the lifetime of the
          galaxies.
    \end{enumerate}
    

% \bibliographystyle{plainnat}
% \bibliography{332}
\end{document}
