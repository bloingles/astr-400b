% Define document class
\documentclass[twocolumn]{aastex631}
\DeclareRobustCommand{\Eqref}[1]{Eq.~\ref{#1}}
\DeclareRobustCommand{\Figref}[1]{Fig.~\ref{#1}}
\DeclareRobustCommand{\Tabref}[1]{Tab.~\ref{#1}}
\DeclareRobustCommand{\Secref}[1]{Sec.~\ref{#1}}
% \usepackage{cuted}
% \usepackage{flushend}
\usepackage{amsmath}
%\usepackage{stfloats}
% \graphicspath{{./figures/}}
\usepackage{float}              % for [H] if you ever need it
\usepackage{dblfloatfix}        % fixes two-column float ordering
\usepackage{placeins}

\begin{document}

% Title
\title{Simulated Velocity Dispersion of Bar Stars in the Milky Way–M31 Merger}

\author[0009-0008-2061-4946]{C.~A.~Burt}
\affiliation{University of Arizona, Department of Astronomy \& Steward Observatory, 933 N.~Cherry Ave., Tucson, AZ 85721, USA}

\correspondingauthor{C.~A.~Burt}
\email{caburt@arizona.edu}

\begin{abstract}

  Bars are stellar structures in spiral galaxies whose torques drive
  angular momentum redistribution that propogate throughout their host
  galaxy. The destruction of a bar in a galaxy can signify the
  transformation of a spiral galaxy into an elliptical galaxy. To
  better understand gravitational interactions in the Local Group,
  \citet{vandermarel:12} produces an N-body simulation of the MW, M31,
  and M33 galaxies. I use this data to explore how the strength of the
  bar evolves, especially as it is impacted by the major galacic
  merger predicted by the simulation. I find a clear dimishing of the
  bar strength at the end of the simulation in M31 and MW. These
  findings demonstrate that merger-driven gravitational forces can
  eliminate bars in massive disks, thereby refining our picture of how
  interactions shape disk galaxy morphology.

\end{abstract}

\section{Introduction}

\textbf{Major mergers} in cosmology refer to the collision and
subsequent consolidation of two spiral galaxies where the ratio of
mass between the more massive galaxy and less massive galaxy does not
exceed 4 \citep{wetzel:16}. They represent one of the most dramatic
processes that can alter the characteristics of a galaxy. The
\textbf{dynamical friction} generated from individual gravitational
force interactions between stars in each galaxy are great enough to
destroy the elongated collection of stars near the center of each
galaxy that fuels the galactic nucleus called the \textbf{stellar bar}
\citep[e.g.,][]{knapen:02}. Concurrently, major mergers destroy the
dense spiral arms composed of young stars that define a \textbf{spiral
  galaxy} and convert it into an \textbf{elliptical galaxy}. which are
identified by their ellipsoidal structure and older stellar population
\citep{hubble:36}. Major mergers encapsulate a broader area of
galactic structure and dynamics and by analyzing the evolution of
stellar bars, we gain insight into the mechanisms that drive the
overall transformation of galaxies \citep[e.g.,][]{vandermarel:01}.

This topic is central to our understanding of the processes by which
galaxies change over time, or \textbf{galactic evolution}. The
disruption or alteration of the bar structure plays a critical role in
the transformation of galactic geometry \citep{wu:18}. After all, the
definition of a galaxy as defined by \citet{willman:12} states ``A
galaxy is a \textbf{gravitationally bound} set of stars whose
properties cannot be explained by a combination of baryons (gas, dust
and stars) and Newton’s laws of gravity'', where ``gravitationally
bound'' describes a system where the gravitational potential energy is
stronger than the potential energy. Bar dynamics, while concerning a
small fraction of the galactic population, exemplifies a region that
demands scrutiny as the definition from \citet{willman:12}
suggests. Furthermore, the presence and nature of bars significantly
affects star formation and the secular evolution of galaxies
\citep{schoenrich:17}. Understanding how mergers affect these
structures helps elucidate the processes that cause barred spiral
galaxies to transition into elliptical galaxies, ultimately shaping
the observable characteristics of galaxies in the universe.

Current research indicates that major galactic mergers prompt
significant changes in spiral galaxies, frequently leading to an
evolution towards an elliptical shape
\citep[e.g.,][]{mutch:11}. Although simulations have successfully
replicated some aspects of this transformation, the details of the
intermediate stages remain poorly understood \citep{berentzen:03}. The
prevailing view is that the bar structure, characteristic of many
spiral galaxies, is significantly disrupted during mergers
\citep[e.g.,][]{mihos:96,mutch:11}. The dynamics of stars within the
bar are disturbed, leading to a chaotic dispersal throughout the
merged galaxy, as shown in \Figref{fig:elmegreen}. Despite a
qualitative understanding of the overall process, there exist gaps in
our knowledge of the precise mechanisms at work.

\begin{figure}[htbp]
    \centering
    \includegraphics[width=\columnwidth]{elmegreen07}
    \caption{Figure 2 from \citet{elmegreen:07}. Peak relative
      amplitude of the Fourier component vs. the normalized radius at
      the peak. The different windows show galactic surveys delineated
      by emission detected \citep[see][]{elmegreen:07}. For all
      surveys, the bars that have higher peak relative amplitudes of
      the Fourier component are longer compared to their galaxy size.}
    \label{fig:elmegreen}
\end{figure}


Despite progress in simulating and observing galactic mergers, several
open questions persist. The exact physical measurements and
properties of bar structures in spiral galaxies are the subject of an
ongoing debate, which greatly influences the evolution of galactic
centers during a merger \citep{rathore:25}. Furthermore, it is partially unclear
how individual perturbations during a merger contribute to the overall
disruption of the bar, and how we can predict the detailed dynamics of
these events \citep{berentzen:03}. Addressing these uncertainties is vital for
developing a more comprehensive model of galaxy evolution post-merger,
ultimately enhancing our understanding of the lifecycle of galaxies.

\begin{figure}[htbp]
  \centering
  \includegraphics[width=\columnwidth]{dehnen23.jpeg}
  \caption{Figure 7 from \citet{dehnen:23}. The visual demonstrates
    one method used to determine the bar size and shape. This figure
    is generated from simulation snapshots, which mirrors the method I
    discuss later in the proposal.}
  \label{fig:dehnen}
\end{figure}

\section{This Project}

Here, I study the influence the major merger predicted between the
Milky Way and Andromeda (M31) galaxies as simulated in
\citet{vandermarel:12} has on the strength of the peak of the Fourier
amplitude in each constituent galaxy.

I use a modified classification for the stellar bar from
\citet{dehnen:23} to determine how force interactions between stars
contribute to the overall disruption of the bar during a major merger.

The destruction of a bar indicates the evolution from a barred spiral
galaxy to an elliptical galaxy. With respect to this, the overall
structure of a galaxy dependent on the strength of a bar if present
and understanding the evolution of the bar is tantamount to
understanding galactic evolution in terms of galactic shape, star
formation, and the ultimate fate of the Local Group.

\section{Methodology}
  \label{methods}
This manuscript uses data from \citet{vandermarel:12}. This is an
optimized N-body simulation containing stars in the disk and bulge of
MW, M31, and M33. An N-body simulation is a simulation in which each
timestep involves a gravitational force calculation between each
particle and all other particles, finding the net acceleration and
applying that to an orbit integration method. \citet{vandermarel:12}
optimizes this method by grouping together weak force interactions
that influence a particle in the same direction and neglecting very
weak force interactions. This simulation models the dynamics of the
Local Group as it experiences a major merger.

To perform any analysis on the barred region of a spiral galaxy, one
must first define the bar. As previously discussed, this remains an
active area of research \citep{berentzen:03}. For this project, I
employ the method depicted in \Figref{fig:dehnen} from
\citet{dehnen:23} due to our analogous methodologies. For this paper, I use
the high-resolution positional disk and bulge star data from
\citet{vandermarel:12} to employ the method from
\citet{dehnen:23}. After identifying the stars located in their host
galaxy's bar, I write a function that identifies close encounters
between the Milky Way and M33 and generates a list of snapshots with
an interval of 16 at ``quiet'' intervals, an interval of 1 surrounding
close encounters and interval sizes of 2, 4, and 8 to provide smooth
transitions. At each snapshot, I calculate the bar strength.

My code will follow a modified version of the calculations from
Appendix B of \citet{dehnen:23}. This first involves separating the
galaxy into major annuli with a population of stars between
$N_\text{min}$ and $N_\text{max}$ such that the maximum and minimum
radius of each annulus are related by the equation
$R_{i,\text{max}}/R_{i,\text{min}} < 10^\Delta$. I then create
overlapping annuli halfway between the $N$ major annuli to create
$2N-1$ annuli in total. I then run a Fourier analysis of each annulus
to find the bar strength $|c_n|$ where $c_n = Me^{-2i\theta}$.
Finally, the code takes the mass-weighted average amplitude and uses
this as the strength of the bar at each snapshot.

% \FloatBarrier
% \begin{figure*}[h!]
%   \centering
%   \includegraphics[width=1.0\textwidth]{bar_histogram}
%   \caption{The bar strength of annuli in each galaxy generated as
%     detailed in \Secref{methods} at four select points between close
%     encounters of MW and M31 bith respect to radial distance from the
%     galactic center of mass of the host galaxy. The total galaxy
%     strength is taken from the mass-weighted average strengths of the
%     annuli. Annuli taken after the first close encounter lose the
%     abundance of stars orbiting at $\sim 20\,\mathrm{kpc}$ away from the
%     galactic center that characterize the bars of MW and M31}
%   \label{fig:annuli}
% \end{figure*}
% \FloatBarrier
\begin{figure*}[htbp]
  \centering
  \includegraphics[width=1.0\textwidth]{bar_histogram}
  \caption{The bar strength of annuli in each galaxy generated as
    detailed in \Secref{methods} at four select points between close
    encounters of MW and M31 bith respect to radial distance from the
    galactic center of mass of the host galaxy. The total galaxy
    strength is taken from the mass-weighted average strengths of the
    annuli. Annuli taken after the first close encounter lose the
    abundance of stars orbiting at $\sim 20\,\mathrm{kpc}$ away from the
    galactic center that characterize the bars of MW and M31}
  \label{fig:annuli}
\end{figure*}

I will generate a plot showing the evolution of the bar strength in MW
and M31 to argue that the strength of the bar is virtually
extinguished as the result of the major merger. At special snapshots
depicting the initial state, final state, and states halfway between
close encounters, I generate a plot of the bar strength with respect
to distance from the galactic nucleus, as well as density contour
plots from a face-down and cylindrical perspective that portray the
bar structure of the Milky Way and M31 as well as their merger
product. These plots allow me to address my proposal.

I expect that stars initially located further from the galactic center
will exhibit higher velocities after dispersion.

\section{Results}

At $t=0.0\,\mathrm{Gyr}$, the strongest annulus in MW is located
$2\,\mathrm{kpc}$ away from the center of MW and M31 is located
$4 \,\mathrm{kpc}$ away from the center of M31 as shown in
\Figref{fig:annuli}. This structure is expected, since the strength of
a bar can be approximated by its radial size
\citep{rathore:25}. Following close encounters, the barred structures
are largely displaced, including very irregular structures close to
the merger of MW and M31. After the merger product of MW and M31
stabilize into an elliptical galaxy, at $t=11.43\,\mathrm{Gyr}$, there
are no prominent spikes in any of the annuli and the strength of the
annuli gradually decrease as distance from the center increases.

Calculating the mass-weighted average of the annuli generates a total
strength of the bar for each galaxy. The evolution of the galactic bar
strength is shown in \Figref{fig:galaxy}. The first snapshot of M31 at
$t=0.0\,\mathrm{Gyr}$ records a bar strength of 0.22, which gradually
decreases until the first close encounter at
$3.93\,\mathrm{Gyr}$. Each close encounter causes extreme fluctuations
in the graph caused by the chaotic gravitational interactions present
during the merger. For times $t \gtrsim 7.5\,\mathrm{Gyr}$, the bar
strength of M31 hovers at 0.09, which is less than half its initial
value. MW demonstrates the same effect to a lesser degree, since the
initial bar strength is much lower. The bar strength of MW starts at
0.14 and ends at 0.07, as shown in the top left panel of
\Figref{fig:annuli} as well.

\begin{figure}[htbp]
  \centering
  \includegraphics[width=\columnwidth]{bar_strength}
  \caption{The measure of the mass-weighted average galaxy bar
    strength as detailed in \Secref{methods} with respect to time
    elapsed from today $t=0.0$. The close encounters between MW and
    M31 at $3.93\,\mathrm{Gyr}$, $5.86\,\mathrm{Gyr}$, and $6.71\,\mathrm{Gyr}$
    (the last of which signifying the merger) are followed by large
    spikes in the bar strength lasting millions of years. For M31, we
    measure that the strength of the bar after the merger at the end
    of the simulation $11.4\,\mathrm{Gyr}$ is roughly 3 times weaker than
    today.}
  \label{fig:galaxy}
\end{figure}


\section{Discussion}

My analysis demonstrated the hypothesized result that for M31 and MW,
the bar strength is weakened and the bar structure is destroyed. This
result is less certain for MW, with a lower initial strength. The bar
structure of both galaxies is shown to deteriorate with time, as the
bottom panels lack the clear bump for stars at radii expected of bar
stars, instead appearing uniform with a gradual decrease at higher
radii.

These results corroborated the results from \citet{elmegreen:07},
whereby barred spiral galaxies frequently lose any barred structure
following a major galactic merger. This structure indicated that the
merger product is an elliptical galaxy as expected
\citep[e.g.,][]{mutch:11}. This simulation provides an explanation

The weakening of the bars reproduces the expected results seen in
\citet{elmegreen:07} that observed galaxies with recent major‐merger
signatures systematically lack strong bars. This result consequently
demonstrates the capabilities of an N-body simulation as a tool to
explain phenomena observed in distant galaxies and predict future
cosmic events. This structure indicated that the
merger product is an elliptical galaxy as expected
\citep[e.g.,][]{mutch:11}.

My analysis focuses the simulation from \citet{vandermarel:12}, which
does not include gas. Therefore, the effects of stellar formation
propogated by the major galactic merger are not considered. In
addition, this project only attempts to use radial distance from
galactic center to define the bar, which is susceptible to chaotic
orbits that arise in the immediate aftermath of close
encounters. Also, random fluctuations in radial distance create noise
in the bar value. This is a product of the fact that the bar is a
structure defined by the geometric configuration of billions of stars,
as opposed to any compact structure.

\section{Conclusions}

Bars are stellar structures in spiral galaxies whose torques drive
angular momentum redistribution that propogate throughout their host
galaxy. The destruction of a bar in a galaxy can signify the
transformation of a spiral galaxy into an elliptical galaxy. To better
understand gravitational interactions in the Local Group,
\citet{vandermarel:12} produces an N-body simulation of the MW, M31,
and M33 galaxies. I use this data to explore how the strength of the
bar evolves, especially as it is impacted by the major galacic merger
predicted by the simulation.

The bars of both galaxies are diminished, with this effect most
prominent in M31.  Both the distribution of stellar annuli and bar
strength of the entire galaxy evolve in patterns that can be explained
through degradation of the barred galactic structure. These results
indicate that stellar encounters during major mergers may destroy the
bar structure and usher in the transfer from a spiral galaxy to an
elliptical galaxy.

This project only considers the radius of stars with respect to the
galactic center of mass. A comprehensive analysis of the 3D geometry
of galaxies would abate the random noise that is otherwise unavoidable
in 1D analysis. Better understanding of the geometry of bar
configurations is needed to understand the precise mechanisms that
take effect during major mergers. In addition, the simulation data
from \citet{vandermarel:12}, while robust, cannot capture the true
complexity of the Local Group. Differences between results using the
low- and high-resolution file can be extrapolated to suggest that
increasing the resolution further may clarify results further in turn.

\section{Acknowledgements}

I would like to thank Dr. Gurtina Besla for her tutelage during this
class (ASTR 400B). Her lectures formed the basis of my knowledge in
this field, forming the backbone of this project. I would also like to
think Himansh Rathore for further elicudating me to the specific topic
of galactic bars. I would also like to thank him for reviewing my code
and submissions and offering helpful suggestions that allowed this
project to come to fruition. Finally, I would like to thank
Dr. Mathieu Renzo for allowing me to utilize his computing hours on
high-performance computing to simulate using the high-resolution stars
and giving my results more clarity.
  
\software{This work made use of the following software packages:
  \texttt{astropy} \citep{astropy:2013, astropy:2018, astropy:2022},
  \texttt{matplotlib} \citep{Hunter:2007}, \texttt{numpy}
  \citep{numpy}, \texttt{python} \citep{python}, and \texttt{scipy}
  \citep{2020SciPy-NMeth, scipy_15366870}. \newline
  Software citation information aggregated using
  \texttt{\href{https://www.tomwagg.com/software-citation-station/}{The
      Software Citation Station}}
  \citep{software-citation-station-paper,
    software-citation-station-zenodo}.}

\bibliography{./research.bib}
\end{document}

%%% Local Variables:
%%% mode: LaTeX
%%% TeX-master: t
%%% End:
